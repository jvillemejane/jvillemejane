%%%%%%%%%%%%%%%%%%%%%%%%%%%%%%%%%%%%%%%%%%
% L1 - Villebon Charpak
%%%%%%%%%%%%%%%%%%%%%%%%%%%%%%%%%%%%%%%%%%
%	Régimes transitoires
%%%%%%%%%%%%%%%%%%%%%%%%%%%%%%%%%%%%%%%%%%
%
% Created by:
%	Julien VILLEMEJANE - 01/jun/2024
% Modified by:
%	
%
%%%%%%%%%%%%%%%%%%%%%%%%%%%%%%%%%%%%%%%%%%
% Professional Newsletter Template
% LaTeX Template
% Version 1.0 (09/03/14)
%
% Created by:
% Bob Kerstetter (https://www.tug.org/texshowcase/) and extensively modified by:
% Vel (vel@latextemplates.com)
% 
% This template has been downloaded from:
% http://www.LaTeXTemplates.com
%
% License:
% CC BY-NC-SA 3.0 (http://creativecommons.org/licenses/by-nc-sa/3.0/)
%
%%%%%%%%%%%%%%%%%%%%%%%%%%%%%%%%%%%%%%%%%

\documentclass[10pt]{article} % The default font size is 10pt; 11pt and 12pt are alternatives

%%%%%%%%%%%%%%%%%%%%%%%%%%%%%%%%%%%%%%%%%
% Professional Newsletter Template
% Structural Definitions File
% Version 1.0 (09/03/14)
%
% Created by:
% Vel (vel@latextemplates.com)
% 
% This file has been downloaded from:
% http://www.LaTeXTemplates.com
%
% License:
% CC BY-NC-SA 3.0 (http://creativecommons.org/licenses/by-nc-sa/3.0/)
%
%%%%%%%%%%%%%%%%%%%%%%%%%%%%%%%%%%%%%%%%%

%----------------------------------------------------------------------------------------
%	REQUIRED PACKAGES
%----------------------------------------------------------------------------------------

\usepackage{graphicx} % Required for including images
\usepackage{microtype} % Improved typography
\usepackage{multicol} % Used for the two-column layout of the document
\usepackage{booktabs} % Required for nice horizontal rules in tables
\usepackage{wrapfig} % Required for in-line images
\usepackage{float} % Required for forcing figures not to float with the [H] parameter

%------------------------------------------------
% Fonts

\usepackage{charter} % Use the Charter font as the main document font
\usepackage{courier} % Use the Courier font for \texttt (monospaced) only
\usepackage[T1]{fontenc} % Use T1 font encoding
\usepackage{lmodern}

%------------------------------------------------
% List Separation

\usepackage{enumitem} % Required to customize the list environments
\setlist{noitemsep,nolistsep} % Remove spacing before, after and within lists for a compact look

%------------------------------------------------
% Figure and Table Caption Styles

\usepackage{caption} % Required for changing caption styles
\captionsetup[table]{labelfont={bf,sf},labelsep=period,justification=justified} % Specify the table caption style
\captionsetup[figure]{labelfont={sf,bf},labelsep=period,justification=justified, font=small} % Specify the figure caption style
\setlength{\abovecaptionskip}{10pt} % Whitespace above captions

%------------------------------------------------
% Spacing Between Paragraphs

\makeatletter
\usepackage{parskip}
\setlength{\parskip}{6pt}
\newcommand{\@minipagerestore}{\setlength{\parskip}{6pt}}
\makeatother

%----------------------------------------------------------------------------------------
%	PAGE MARGINS AND SPACINGS
%----------------------------------------------------------------------------------------

\textwidth = 7 in % Text width
\textheight = 10 in % Text height
\oddsidemargin = -18pt % Left side margin on odd pages
\evensidemargin = -18pt % Left side margin on even pages
\topmargin = -36pt % Top margin
\headheight = 0pt % Remove the header by setting its space to 0
\headsep = 0pt % Remove the space between the header and top of the page
\parskip = 4pt % Space between paragraph
\parindent = 0.0in % Paragraph indentation
\pagestyle{empty} % Disable page numbering

%----------------------------------------------------------------------------------------
%	COLORS
%----------------------------------------------------------------------------------------

\usepackage[dvipsnames,svgnames]{xcolor} % Required to specify custom colors

\definecolor{altncolor}{rgb}{.8,0,0} % Dark red
%\definecolor{altncolor}{rgb}{.2,.4,.8} % Dark blue
%\definecolor{altncolor}{rgb}{.84,.16,.16} % Red

\usepackage[colorlinks=true, linkcolor=altncolor, anchorcolor=altncolor, citecolor=altncolor, filecolor=altncolor, menucolor=altncolor, urlcolor=altncolor]{hyperref} % Use the color defined above for all links

\definecolor{darkblue}{rgb}{0.0, 0.2, 0.6}

%----------------------------------------------------------------------------------------
%	MISC.
%----------------------------------------------------------------------------------------


\usepackage{amsmath}
\usepackage{circuitikz, verbatim}
% http://math.et.info.free.fr/TikZ/bdd/TikZ-Impatient.pdf
\usetikzlibrary{backgrounds}
\ctikzset{}
\usepackage{pgfplots}

%----------------------------------------------------------------------------------------
%	BOX STYLES
%----------------------------------------------------------------------------------------

\usepackage[framemethod=TikZ]{mdframed}% Required for creating boxes
\mdfdefinestyle{sidebar}{
    linecolor=black, % Outer line color
    outerlinewidth=0.5pt, % Outer line width
    roundcorner=0pt, % Amount of corner rounding
    innertopmargin=10pt, % Top margin
    innerbottommargin=10pt, % Bottom margin
    innerrightmargin=10pt, % Right margin
    innerleftmargin=10pt, % Left margin
    backgroundcolor=white, % Box background color
    frametitlealignment=\centering,
    frametitlebackgroundcolor=gray!20, % Title background color
    frametitlerule=false, % Title rule - true or false
    frametitlerulecolor=white, % Title rule color
    frametitlerulewidth=0.5pt, % Title rule width
    frametitlefont=\Large\bfseries, % Title heading font specification
    font=\small
}

\mdfdefinestyle{aavbox}{
    linecolor=black, % Outer line color
    outerlinewidth=0.2pt, % Outer line width
    roundcorner=5pt, % Amount of corner rounding
    innertopmargin=7pt, % Top margin
    innerbottommargin=7pt, % Bottom margin
    innerrightmargin=7pt, % Right margin
    innerleftmargin=7pt, % Left margin
    backgroundcolor=gray!10, % Box background color
    frametitlealignment=\centering,
    frametitlebackgroundcolor=gray!30, % Title background color
    frametitlerule=false, % Title rule - true or false
    frametitlerulecolor=white, % Title rule color
    frametitlerulewidth=0.2pt, % Title rule width
    frametitlefont=\Large\bfseries % Title heading font specification
}

%----------------------------------------------------------------------------------------
%	HEADING STYLE
%----------------------------------------------------------------------------------------

\newcommand{\heading}[2]{ % Define the \heading command
\vspace{#2} % White space above the heading
{\begin{center}\Large\textbf{#1}\end{center}} % The heading style
\vspace{#2} % White space below the heading
}


\usepackage{titlesec}

%% SECTIONS
\titleformat{\section} % command
[display]
{\bfseries\Large\color{darkblue}} % format
{} % label
{0.5ex} % sep
{
    \vspace{-5ex}
    \rule{\textwidth}{1pt}
    \vspace{1ex}
    \thesection. } % before-code
[
\vspace{-3ex}%
\rule{\textwidth}{0.3pt}
] % after-code

\titleformat{name=\section, numberless} % command
[display]
{\bfseries\Large\color{darkblue}} % format
{} % label
{0.5ex} % sep
{
    \vspace{-2ex}
    \rule{\textwidth}{1pt}
    \vspace{0.2ex}
} % before-code
[
\vspace{-3ex}%
\rule{\textwidth}{0.5pt}
] % after-code

%% SUBSECTION
\titleformat{\subsection} % command
{\bfseries\large\itshape\color{darkblue}} % format
{} % label
{0.1\textwidth} % sep
{
    \vspace{-0.4em}
    \thesubsection. } % before-code
[
\vspace{-0.7em}
\hspace{0.1\textwidth}
\rule{0.9\textwidth}{0.3pt}
] % after-code

\titleformat*{\subsubsection}{\itshape\bfseries\color{darkblue}}


%----------------------------------------------------------------------------------------
%	THEOREMS
%----------------------------------------------------------------------------------------

\mdfdefinestyle{definitionstyle}{%
linecolor=purple,linewidth=2,%
frametitlerule=true,%
frametitlebackgroundcolor=purple,
frametitlefontcolor=white,
frametitlefont=\large\bfseries,
innertopmargin=0.5em,
skipabove=1em,
roundcorner=5pt,
}

 % Include the document which specifies all packages and structural customizations for this template
\usepackage{amsmath}

%----------------------------------------------------------------------------------------
%	DOCUMENT INFORMATIONS
%----------------------------------------------------------------------------------------
\def\module{Régimes Transitoires}
\def\submodule{Régimes Transitoires}
\def\moduleSmall{Rég. Tra.}
\def\year{2023-2024}
\def\problem{Etude d'un circuit RC}
\def\problemName{Mettre en oeuvre un circuit RC et le modéliser}

\def\validation{}

\def\scheduleCM{1}
\def\scheduleTD{0}
\def\scheduleTP{2}

\def\workingTeam{Par binôme}

\def\workingSpecial{}

\def\keywords{Etat stable; Régime transitoire; Constante de temps; Circuit RC}


\begin{document}
%----------------------------------------------------------------------------------------
%	HEADER IMAGE
%----------------------------------------------------------------------------------------

\begin{figure}[H]
\centering\includegraphics[width=0.5\linewidth]{./_assets/latex/logo_ivc.png}
\end{figure}

%----------------------------------------------------------------------------------------
%	SIDEBAR - FIRST PAGE
%----------------------------------------------------------------------------------------

\begin{minipage}[t]{.33\linewidth} % Mini page taking up 30% of the actual page
\begin{mdframed}[style=sidebar,frametitle={\module}] % Sidebar box

%-----------------------------------------------------------
%	DOCUMENT DESCRIPTION
\begin{center}

\textit{\large \centering \year}
\end{center}


\centerline {\rule{.70\linewidth}{.25pt}} % Horizontal line

\begin{center}
	\textit{\large \moduleSmall}
\end{center}

\centerline {\rule{.70\linewidth}{.25pt}} % Horizontal line

\begin{center}
	\textbf{\problem} ( \validation )
\end{center}

\centerline {\rule{.70\linewidth}{.25pt}} % Horizontal line

%-----------------------------------------------------------

\textbf{Concepts étudiés}

\begin{itemize}
\item[\textsc{\scriptsize [Phys]}] Régime Transitoire
\item[\textsc{\scriptsize [Math]}] Equation différentielle
\item[\textsc{\scriptsize [Ing]}] Instrumentation et mesures
\item[\textsc{\scriptsize [Ing]}] Réalisation d'un circuit
\end{itemize}

\centerline {\rule{.70\linewidth}{.25pt}} % Horizontal line

%-----------------------------------------------------------

\textbf{Mots clefs}

\keywords

\centerline {\rule{.70\linewidth}{.25pt}} % Horizontal line

%-----------------------------------------------------------

\textbf{Sessions}

\begin{itemize}
\item[\textbf{\scheduleCM}] Cours(s) - 1h30
\item[\textbf{\scheduleTD}] TD(s) - 1h30
\item[\textbf{\scheduleTP}] TP(s) - 3h00
\end{itemize}

\centerline {\rule{.70\linewidth}{.25pt}} % Horizontal line

{\large Travail}

\textbf{\workingTeam}

\textbf{\workingSpecial}

%-----------------------------------------------------------
\end{mdframed}


\begin{minipage}[t]{.95\linewidth}
\textbf{Institut Villebon - Georges Charpak}\\
\href{https://www.villebon-charpak.fr/}{https://www.villebon-charpak.fr/}


\end{minipage}\hfill % End the sidebar mini page 
% NO SPACE BETWEEN THE END OF SIDEBAR AND BEGIN OF MAIN PART
%----------------------------------------------------------------------------------------
%	MAIN BODY - FIRST PAGE
%----------------------------------------------------------------------------------------
%
\begin{minipage}[t]{.65\linewidth} % Mini page taking up 65% of the actual page

\hypertarget{context}{\heading{\huge \problemName}{6pt}} % \hypertarget provides a label to reference using \hyperlink{label}{link text}

\centerline {\rule{.70\linewidth}{.25pt}} % Horizontal line

%% Short introduction 
La plupart du temps, la \textbf{transition entre deux états stables} n'est pas instantané. 

La définition d'un régime transitoire en physique est la suivante :

\begin{mdframed}[style=aavbox,frametitle={Régime transitoire}]
\textbf{Régime d'évolution} d'un système qui n'a pas atteint un \textbf{état stable} (ou \textbf{régime permanent}).
\end{mdframed}

La durée caractéristique d'un régime transitoire est appelée \textbf{temps de relaxation} ou \textbf{constante de temps} de ce système.

%%

\bigskip

%----------------------------------------------------------------------------------------
%	IN-TEXT BOX / Intended learning outcomes
%----------------------------------------------------------------------------------------

\begin{mdframed}[style=aavbox,frametitle={Acquis d'Apprentissage Visés}]

A la fin de ce module, vous serez capable de :

\begin{enumerate}
\item \textbf{Modéliser un phénomène transitoire} du premier ordre
\item \textbf{Mettre en oeuvre un protocole de mesure} du temps de réponse d'un circuit RC
\item \textbf{Rédiger un compte-rendu}
\end{enumerate}
\end{mdframed}

%-----------------------------------------------------------

\hypertarget{exemples}{\heading{Exemples}{6pt}} % \hypertarget provides a label to reference using \hyperlink{label}{link text}

Il existe des régimes transitoires dans de nombreux domaines. Voici quelques exemples :

\begin{itemize}
	\item Mise en chauffe d'un four, d'une pièce...
	\item Atteinte d'une vitesse d'avance (transports, moteurs...), d'une position (ascenceur, imprimante...)
	\item Réactions chimiques (ions en solution acqueuse...)
	\item Charge et décharge d'un condensateur à travers une résistance	
\end{itemize}


\medskip

%-----------------------------------------------------------

\hypertarget{etude}{\heading{Circuit RC}{6pt}} % \hypertarget provides a label to reference using \hyperlink{label}{link text}

Dans le cadre de ce module, nous nous intéresserons à l'étude de la charge (et la décharge) d'un condensateur à travers une résistance (circuit suivant).

CIRCUIT RC

\medskip

\end{minipage} % End the main body - first page mini page

%----------------------------------------------------------------------------------------
%	MAIN BODY - SECOND PAGE
%----------------------------------------------------------------------------------------
\newpage
Dans la suite de ce module, nous allons nous intéressés au montage suivant :

CIRCUIT RC

\begin{circuitikz}
	\draw (1,0) to [short, *-] (3,0)
		to[R=$R_{2}$, -*, i<_=$I_2$] (3,2)
		to[R=$R_{1}$, -] (3,4)
		to [short, -*, i<_=$I$] (0,4);
	\draw (3,0) to[short, -o] (4,0);
	\draw[dashed] (4,0) to[short, -] (5.5,0) 
		to[R=$R_L$, -] (5.5,2)
		to[short, -] (4,2);
	\draw (3,2) to[short, -o, i=$I_S$] (4,2);
	
	% fleche
	\draw (0,0.5) edge[->] (0,3.5);
	\node (Ein) at (-1,2.25){$V_E$};

	\draw (0,0) to [short, *-] (1,0)
		node[ground](GND){};
	% fleche
	\draw (3.5,0.3) edge[->, green!40!black] (3.5,1.7); \node[text=green!40!black] (US) at (4,1){$V_S$};
\end{circuitikz}

\centerline {\rule{.70\linewidth}{.5pt}} % Horizontal line

%-----------------------------------------------------------

\hypertarget{equations}{\heading{Mise en équation}{6pt}} % \hypertarget provides a label to reference using \hyperlink{label}{link text}

\centerline {\rule{.70\linewidth}{.25pt}} % Horizontal line
%-----------------------------------
\textbf{Loi des mailles}

Si on s'intéresse au circuit précédent, on s'aperçoit qu'il existe une seule maille principale. Il est alors possible d'utiliser la loi des mailles.


\begin{mdframed}[style=aavbox,frametitle={Rappel de la loi des mailles (loi de Kirchhoff)}]

Dans une maille quelconque d'un réseau, la somme algébrique des différences de potentiel le long de la maille est constamment nulle.

Cette loi découle de l'additivité des différences de potentiel entre deux points. La différence de potentiels entre a et b, aussi appelée tension, est exprimée par l'équation suivante : $U_{ab} = V_a - V_b$
\end{mdframed}

Dans notre cas, on trouve :

$$\boxed{E - u_R - u_C = 0}$$

\medskip

Comme il n'y a qu'une seule maille, il n'y a pas de noeuds dans ce circuit. Le courant qui traverse la résistance R est le même que le courant traversant le condensateur C.

\centerline {\rule{.70\linewidth}{.25pt}} % Horizontal line
%-----------------------------------
\textbf{Capacité d'un condensateur}

La \textbf{capacité} d'un condensateur représente la \textbf{quantité de charges électriques} qu'il accumule pour une différence de potentiel donnée.

Elle est définie par : $$C = \frac{q}{u_C}$$ où $u_C$ est la différence de potentiel aux bornes du condensateur en volts (V), $q$ la quantité de charges électriques en coulombs (C) et $C$ la capacité en farads (F).

\centerline {\rule{.70\linewidth}{.25pt}} % Horizontal line
%-----------------------------------
\textbf{Courant électrique et charges}

Un \textbf{courant électrique} est un \textbf{mouvement de charges électriques} (généralement des électrons) au sein d'un matériau conducteur.

L'\textbf{intensité d'un courant}, en ampères (A) décrit la quantité de charges électriques qui transitent à travers une surface donnée. On peut voir ce courant comme un \textbf{débit d'électrons}.

On remonte à l'intensité par la loi suivante :

$$i(t) = \frac{{\rm d}q(t)}{{\rm d}t}$$


\centerline {\rule{.70\linewidth}{.25pt}} % Horizontal line
%-----------------------------------
\textbf{Lien entre C et $u_C$}

Ainsi, il est possible de trouver le lien entre le courant qui traverse un condensateur et la valeur de sa capacité :

$$q(t) = C \cdot u_C(t)$$

$$i(t) = \frac{{\rm d}q(t)}{{\rm d}t} = \frac{{\rm d}(C \cdot u_C(t))}{{\rm d}t}$$

On suppose que la capacité C est constante, ainsi :

$$\boxed{i(t) = C \cdot \frac{{\rm d}u_C(t)}{{\rm d}t}}$$


\centerline {\rule{.70\linewidth}{.25pt}} % Horizontal line
%-----------------------------------
\textbf{Loi d'Ohm}

La loi d'Ohm aux bornes de la résistance R permet d'écrire : $$\boxed{u_R = R \cdot i(t)}$$

\centerline {\rule{.70\linewidth}{.25pt}} % Horizontal line
%-----------------------------------
\textbf{Montage complet}

En reprenant l'ensemble des équations précédentes, on obtient :

$$E = R \cdot i(t) + u_C(t)$$

$$\boxed{E = R \cdot C \cdot \frac{{\rm d}u_C(t)}{{\rm d}t} + u_C(t)}$$

Il s'agit d'une équation différentielle du premier ordre.

\medskip

Il existe alors des outils mathématiques pour la résoudre, dans un cas particulier.

Dans notre cas, on supposera qu'à l'instant $t=0$, la tension $u_C(t) = 0$ et qu'on ferme l'interrupteur K. Ainsi, la tension $u_E(t)$ passe instantanément de 0 à une tension positive $E$.

On obtient alors la fonction suivante pour $u_C(t)$ (qui est également la tension de sortie de notre système) :

$$\boxed{u_C(t) = E \cdot (1 - \exp{\frac{-t}{R \cdot C}}}$$

Par la suite, on appelera $\tau = R \cdot C$ (\textit{tau}) la \textbf{constante de temps} du circuit.


\centerline {\rule{.70\linewidth}{.5pt}} % Horizontal line

\hypertarget{verification}{\heading{Vérification mathématique}{6pt}} % \hypertarget provides a label to reference using \hyperlink{label}{link text}

On peut vérifier que la fonction obtenue précédemment répond bien à une solution de l'équation différentielle du circuit et que d'autres ne peuvent pas être solution de cette loi d'évolution du circuit.


\centerline {\rule{.70\linewidth}{.25pt}} % Horizontal line
%-----------------------------------
\textbf{$u_C(t) = A$ (constante)}

On a alors ${\rm d}u_C(t)/{\rm d}t = 0$

Et : $R \cdot C \cdot \frac{{\rm d}u_C(t)}{{\rm d}t} + u_C(t) = A$

Cela ne fonctionne que si $A = E$, c'est à dire si la capacité est déjà chargée.

\textbf{$u_C(t) = B \cdot t + A$ (loi affine)}

On a alors ${\rm d}u_C(t)/{\rm d}t = B$

Et : $R \cdot C \cdot \frac{{\rm d}u_C(t)}{{\rm d}t} + u_C(t) = A + R \cdot C \cdot B$

Cette relation donne une valeur différente de $E$.

\textbf{$u_C(t) = E \cdot (1 - \exp{\frac{-t}{R \cdot C}}$ (exponentielle)}

On a alors ${\rm d}u_C(t)/{\rm d}t = -E/(R \cdot C) \cdot \exp{\frac{-t}{R \cdot C}}$

Et : $R \cdot C \cdot \frac{{\rm d}u_C(t)}{{\rm d}t} + u_C(t) = E \cdot (1 - \exp{\frac{-t}{R \cdot C}} - E/(R \cdot C) \cdot \exp{\frac{-t}{R \cdot C}}$

On obtient alors : $R \cdot C \cdot \frac{{\rm d}u_C(t)}{{\rm d}t} + u_C(t) = E$

\centerline {\rule{.70\linewidth}{.5pt}} % Horizontal line

\hypertarget{realisation}{\heading{Mise en oeuvre}{6pt}} % \hypertarget provides a label to reference using \hyperlink{label}{link text}





\centerline {\rule{.70\linewidth}{.25pt}} % Horizontal line


\end{document} 